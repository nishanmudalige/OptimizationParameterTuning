\documentclass{article}
\usepackage{graphicx}
\usepackage{tabularx}
\usepackage{array} % for \newcolumntype and \arraybackslash (optional but nice)
\usepackage{float}


% Optional: nicer spacing inside rows
\renewcommand{\arraystretch}{1.15}
% Optional: reduce horizontal padding a bit
\setlength{\tabcolsep}{8pt}

% Optional: left-aligned X column with proper \\ handling
\newcolumntype{Y}{>{\raggedright\arraybackslash}X}

\title{STA378 Final Report}
\author{Felix Gao}
\date{September 2025}

\begin{document}
\maketitle

\section{Variables Descriptions}
\subsection*{Data Table Columns}

\begin{table}[H]
\centering
\begin{tabularx}{\textwidth}{|l|l|Y|}
\hline
\textbf{Column} & \textbf{Type} & \textbf{Description} \\
\hline
\texttt{status} & Symbol &
\textbf{first\_order}: solver successfully reached a first-order stationary point; 
\textbf{max\_time}: solver couldn't solve the problem within the time limit; 
\textbf{unbounded}: the minimum of the objective function goes to $-\infty$. \\
\hline
\texttt{name} & String & Identifier for the optimization problem. \\
\hline
\texttt{solver} & String & Name of the solver/algorithm applied. \\
\hline
\texttt{mem} & Int & Hyperparameter of the solver. \\
\hline
\texttt{nvar} & Int & Number of decision variables in the problem. \\
\hline
\texttt{time} & Float64 & Total runtime in seconds (s). \\
\hline
\texttt{memory} & Float64 & Total memory in megabytes (MB). \\
\hline
\texttt{num\_iter} & Int & Total number of iterations performed. \\
\hline
\texttt{nvmops} & Int & Number of vector–matrix products. \\
\hline
\texttt{neval\_obj} & Int & Number of objective function evaluations. \\
\hline
\texttt{init\_eval\_obj\_time} & Float64 & Time (seconds) for the initial objective evaluation. \\
\hline
\texttt{neval\_grad} & Int & Number of gradient evaluations. \\
\hline
\texttt{init\_eval\_grad\_time} & Float64 & Time (seconds) for the initial gradient evaluation. \\
\hline
\texttt{is\_init\_run} & Bool & Indicates whether this is the initial run. \\
\hline
\end{tabularx}
\caption{Description of columns in the solver benchmark data table.}
\end{table}

\section{}



\end{document}