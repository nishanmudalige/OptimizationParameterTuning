\documentclass{article}
\usepackage{graphicx} % Required for inserting images

\title{STA378 Documentations}
\author{Felix Gao}
\date{September 2025}

\begin{document}

\maketitle

\section{Variables Descriptions}
\subsection*{Data Table Columns}

\begin{table}[h!]
\centering
\begin{tabular}{|l|l|p{9cm}|}
\hline
\textbf{Column} & \textbf{Type} & \textbf{Description} \\
\hline
\texttt{status} & Symbol   & Outcome of the solver run (e.g., \texttt{first\_order}, \texttt{max\_time}, \texttt{unbounded}). \\
\hline
\texttt{name} & String   & Identifier for the optimization problem. \\
\hline
\texttt{solver} & String   & Name of the solver/algorithm applied. \\
\hline
\texttt{mem} & Int   & Hyperparameter of the solver. \\
\hline
\texttt{nvar} & Int   & Number of decision variables in the problem. \\
\hline
\texttt{time} & Float64   & Total runtime in seconds (s). \\
\hline
\texttt{memory} & Float64   & Total memory in megabytes (MB). \\
\hline
\texttt{num\_iter} & Int   & Total number of iterations performed. \\
\hline
\texttt{nvmops} & Int   & Number of vector/matrix operations executed. \\
\hline
\texttt{neval\_obj} & Int   & Number of objective function evaluations. \\
\hline
\texttt{init\_eval\_obj\_time} & Float64   & Time (seconds) for the initial objective evaluation. \\
\hline
\texttt{neval\_grad} & Int   & Number of gradient evaluations. \\
\hline
\texttt{init\_eval\_grad\_time} & Float64   & Time (seconds) for the initial gradient evaluation. \\
\hline
\end{tabular}
\caption{Description of columns in the solver benchmark data table.}
\end{table}


\end{document}
